\documentclass{beamer}
\usepackage[utf8]{inputenc}
\usepackage{babel}
\usepackage{xcolor}
\usepackage{AxiCommands}

\usetheme{Darmstadt}

\title{Axi Design: Language Features}
\author{Wojciech Kołowski}
\date{}

\begin{document}

\frame{\titlepage}

\section{Totality checking}

\begin{frame}{What is a function, classically?}

In classical mathematics, a relation $R$ between $A$ and $B$ is called a function when it is:

\begin{itemize}
  \item Right-unique (also called deterministic): for all $a : A$ and $b_1, b_2 : B$, if $R(a, b_1)$ and $R(a, b_2)$, then $b_1 = b_2$.
  \item Right-total (also called total): for all $a : A$, there exists $b : B$ such that $R(a, b)$.
\end{itemize}

\end{frame}

\begin{frame}{What is a function, constructively?}

In constructive mathematics, TODO
\end{frame}

\begin{frame}{What is totality checking?}

Totality checking is the following task: we are given the definition of a well-typed function $f : A \to B$ and we need to tell whether it is total, i.e. whether $\forall a : A, \exists b : B, f(a) = b$. Spelled out in words: for all $a$ of type $A$, there exists $b$ of type $B$ such that $f(a)$ equals $b$.

\end{frame}

\begin{frame}{Why should we care?}

Why should we care about totality checking? TODO
\end{frame}

\begin{frame}{Why should we care?}

We already know that all functions must terminate for proof checking to be decidable. However, this isn't termination checking's raison d'être.

\vspace{2em}

Neither has it anything to do with gas, or even what happens at runtime at all. After all, the universe is going to last a finite amount of time, so all functions will eventually terminate, right? Wrong.

\vspace{2em}

Termination is first and foremost a logical notion. It's evil twin, non-termination, manifests itself most strikingly not at runtime, where it cannot be observed at all (because it would take forever), but in the logic -- TODO

\end{frame}

\begin{frame}{Termination checking is undecidable}

In general, termination checking is undecidable -- it means literally solving the Halting Problem! That's hard!

\vspace{2em}

\end{frame}

\end{document}