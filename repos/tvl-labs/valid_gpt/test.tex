\documentclass[a4paper]{article}
\usepackage{mathtools}
\begin{document}
                                Here's the natural language translation of the given formal proof:

We will show that the negation of $F$ is a logical consequence of the following three premises:

$p1: (A \lor B \Rightarrow C \land D)$
$p2: (C \lor E \Rightarrow \neg F \land G)$
$p3: (F \lor H \Rightarrow A \land I)$

We will derive $\neg F$ by contradiction. Let's assume $F$ and show that this leads to a contradiction.

Starting with $F$, we can reason as follows:

$F$
$\Rightarrow (F \lor H)$ (by the alternate rule)
$\Rightarrow (A \land I)$ (by premise $p3$)
$\Rightarrow A$ (by left-and)
$\Rightarrow (A \lor B)$ (by the alternate rule)
$\Rightarrow (C \land D)$ (by premise $p1$)
$\Rightarrow C$ (by left-and)
$\Rightarrow (C \lor E)$ (by the alternate rule)
$\Rightarrow (\neg F \land G)$ (by premise $p2$)
$\Rightarrow \neg F$ (by left-and)

This final result, $\neg F$, contradicts our initial assumption $F$. Therefore, we have derived a contradiction from the assumption $F$.

Thus, by the principle of proof by contradiction, we can conclude $\neg F$.

This proves that $\neg F$ is indeed a logical consequence of the given premises.
\end{document}
                                